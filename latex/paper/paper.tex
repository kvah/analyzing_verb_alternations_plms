% This must be in the first 5 lines to tell arXiv to use pdfLaTeX, which is strongly recommended.
\pdfoutput=1
% In particular, the hyperref package requires pdfLaTeX in order to break URLs across lines.

\documentclass[11pt]{article}

% Remove the "review" option to generate the final version.
\usepackage{acl}

% Standard package includes
\usepackage{times}
\usepackage{latexsym}

% For proper rendering and hyphenation of words containing Latin characters (including in bib files)
\usepackage[T1]{fontenc}
% For Vietnamese characters
% \usepackage[T5]{fontenc}
% See https://www.latex-project.org/help/documentation/encguide.pdf for other character sets

% This assumes your files are encoded as UTF8
\usepackage[utf8]{inputenc}

% This is not strictly necessary, and may be commented out,
% but it will improve the layout of the manuscript,
% and will typically save some space.
\usepackage{microtype}

% Nicer enumerations:
\usepackage{enumitem}

% provides the \ex command to make linguistics-style numbered examples
\usepackage{gb4e}
% recommended by https://tex.stackexchange.com/questions/325621/gb4e-package-causing-capacity-errors
\noautomath

% much prettier tables:
\usepackage{booktabs}
\usepackage[singlelinecheck=false]{caption} 

\newcommand{\lookout}[1]{\textcolor{blue}{\textbf{#1}}}


\title{Probing for Understanding of English Verb Classes and Alternations in Large Language Models}

% Author information can be set in various styles:
% For several authors from the same institution:
% \author{Author 1 \and ... \and Author n \\
%         Address line \\ ... \\ Address line}
% if the names do not fit well on one line use
%         Author 1 \\ {\bf Author 2} \\ ... \\ {\bf Author n} \\
% For authors from different institutions:
% \author{Author 1 \\ Address line \\  ... \\ Address line
%         \And  ... \And
%         Author n \\ Address line \\ ... \\ Address line}
% To start a seperate ``row'' of authors use \AND, as in
% \author{Author 1 \\ Address line \\  ... \\ Address line
%         \AND
%         Author 2 \\ Address line \\ ... \\ Address line \And
%         Author 3 \\ Address line \\ ... \\ Address line}

\author{James V.~Bruno, Jiayu Han, David K.~Yi, Peter Zukerman \thanks{ The authors' names appear in alphabetical order.}\\
The University of Washington\\
  \texttt{\{jbruno, jyhan126, davidyi6, pzuk\}@uw.edu}}

\begin{document}
\maketitle
\begin{abstract}
Abstract goes here, man.
\end{abstract}

\section{Introduction}

Cite these guys to make sure the bibliography works.  \cite{kann-etal-2019-verb}

\section{Related Work}

\section{Data}
\section{Method}
\section{Results}

The results of experiment $1$ appear in Table~\ref{tab:word-embeddings-results}.


\begin{table*}
\begin{tabular}{lrrr}
\toprule
{} &  \multicolumn{1}{c}{Baseline} &           &    \\
{} &  \multicolumn{1}{c}{Accuracy} &  Accuracy & MCC \\
\midrule
\textbf{Causative-Inchoative} &        &        &        \\
Inchoative           &  0.589 &  0.847 &  0.683 \\
Caustive             &  0.589 &  0.847 &  0.683 \\
\addlinespace[6pt]
\textbf{Dative}      &        &        &        \\
Both                 &  0.907 &  0.928 &  0.523 \\
Preposition          &  0.946 &  0.964 &  0.574 \\
Double-Object        &  0.925 &  0.941 &  0.470 \\
\addlinespace[6pt]
\textbf{Spray-Load}  &        &        &        \\
Both                 &  0.915 &  0.936 &  0.480 \\
\textit{With}        &  0.790 &  0.875 &  0.603 \\
Locative             &  0.834 &  0.878 &  0.506 \\
\addlinespace[6pt]
\textbf{\textit{There}-insertion} & &  &        \\
No-there             &  0.664 &  0.779 &  0.492 \\
\textit{There}       &  0.664 &  0.779 &  0.492 \\
\addlinespace[6pt]
\textbf{Understood Object}  &        &        &        \\
Reflexive            &  0.833 &  0.864 &  0.467 \\
Non-reflexive        &  0.975 &  0.984 &  0.615 \\
\bottomrule
\end{tabular}
\caption{Accuracy and Matthews Correlation Coefficients for Syntactic Frame Predictions from Static Word Embeddings from Bert.}
\label{tab:word-embeddings-results}
\end{table*}

\section{Discussion}
\section{Conclusion}

% Entries for the entire Anthology, followed by custom entries
\bibliography{paper}

\appendix

\section{Example Appendix}
\label{sec:appendix}

This is an appendix.

\end{document}
